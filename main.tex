\documentclass[12pt]{article}
\usepackage{braket}
\usepackage{physics}
\usepackage{graphicx}
\usepackage{times}
\usepackage[export]{adjustbox}
\usepackage{listings}
\usepackage{mathcomp}
\usepackage{hyperref}
\usepackage{bm,amsmath}
\usepackage{float}
\usepackage{indentfirst}
\usepackage{bigints}
\usepackage{listings}
\usepackage{color}
\hypersetup{
colorlinks=true,
linkcolor=blue,
filecolor=magenta,
urlcolor=cyan,
pdftitle={Overleaf Example},
pdfpagemode=FullScreen,
}
\definecolor{dkgreen}{rgb}{0,0.6,0}
\definecolor{gray}{rgb}{0.5,0.5,0.5}
\definecolor{mauve}{rgb}{0.58,0,0.82}
\lstset{frame=tb,
language=Python,
aboveskip=3mm,
belowskip=3mm,
stepnumber = 1,
showstringspaces=false,
columns=flexible,
basicstyle={\small\ttfamily},
numbers=left,
numberstyle=\color{gray},
keywordstyle=\color{blue},
commentstyle=\color{dkgreen},
stringstyle=\color{mauve},
breaklines=true,
breakatwhitespace=true,
tabsize=3
}
\numberwithin{equation}{section}

\title{Notes of Quantum Theory of Fields by S.Weinberg}
\author{Ting-Kai Hsu}
\date{\today}

\begin{document}
\maketitle
\tableofcontents
\section{Historical Introduction}
\section{Relativistic Quantum Mechanics}
\subsection{Symmetries}
If the observations of possible experiments of a system do not change under a transformation, then the transformation is defined to be \textit{symmetry transformation}.
Saying in another way, observations under symmetry transformation can be seen as different observers look at the \textit{same} system, and they must find the same probabilities
\begin{equation}
    P(\mathcal{R}\rightarrow\mathcal{R}_n) = P(\mathcal{R'}\rightarrow\mathcal{R}'_n)
\end{equation} 
This is the only condition for a transformation to be a symmetry.

\end{document}