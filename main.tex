\documentclass[12pt]{article}
\usepackage{braket}
\usepackage{physics}
\usepackage{graphicx}
\usepackage{times}
\usepackage[export]{adjustbox}
\usepackage{listings}
\usepackage{mathcomp}
\usepackage{hyperref}
\usepackage{bm,amsmath}
\usepackage{float}
\usepackage{indentfirst}
\usepackage{bigints}
\usepackage{listings}
\usepackage{color}
\hypersetup{
colorlinks=true,
linkcolor=blue,
filecolor=magenta,
urlcolor=cyan,
pdftitle={Overleaf Example},
pdfpagemode=FullScreen,
}
\definecolor{dkgreen}{rgb}{0,0.6,0}
\definecolor{gray}{rgb}{0.5,0.5,0.5}
\definecolor{mauve}{rgb}{0.58,0,0.82}
\lstset{frame=tb,
language=Python,
aboveskip=3mm,
belowskip=3mm,
stepnumber = 1,
showstringspaces=false,
columns=flexible,
basicstyle={\small\ttfamily},
numbers=left,
numberstyle=\color{gray},
keywordstyle=\color{blue},
commentstyle=\color{dkgreen},
stringstyle=\color{mauve},
breaklines=true,
breakatwhitespace=true,
tabsize=3
}
\numberwithin{equation}{subsection}

\title{Quantum Field Theory Problems}
\author{Ting-Kai Hsu}
\date{\today}

\begin{document}
\maketitle
\tableofcontents
\section{T P Symmetries}
In S.Weinberg famous textbook about quantum field theory\cite{Weinberg_1995} section 2.6, he discusses parity$\mathcal{P}$ and time inversion$\mathcal{T}$. 
\footnote{I use different notation with Weinberg.}
\begin{equation}
    \begin{split}
        \mathcal{P}^{\mu}_{\,\,\,\,\nu} = \begin{pmatrix}
            1&0&0&0\\
            0&-1&0&0\\
            0&0&-1&0\\
            0&0&0&-1\\
        \end{pmatrix}\\
        \mathcal{T}_{\,\,\,\,\nu}^{\mu} = \begin{pmatrix}
            -1&0&0&0\\
            0&1&0&0\\
            0&0&1&0\\
            0&0&0&1\\
        \end{pmatrix}
    \end{split}
\end{equation}
The operators of $\mathcal{P}$ and $\mathcal{T}$ are believed to be
\begin{equation}
    \begin{split}
        \text{P} \equiv U(\mathcal{P}, 0)\\
        \text{T} \equiv U(\mathcal{T}, 0)
    \end{split}
\end{equation}
He seems to define a new notation that corresponds to not only Lorentz transformation and translation but also parity and time inversion.
The operators of Poincar\'e algebra would transform according to the following law,
\begin{equation}
    \begin{split}
        \text{P}U(\Lambda, a)\text{P}^{-1} = U(\mathcal{P}\Lambda\mathcal{P}^{-1}, \mathcal{P}a)\\
        \text{T}U(\Lambda, a)\text{T}^{-1} = U(\mathcal{T}\Lambda\mathcal{T}^{-1}, \mathcal{T}a)
    \end{split}
\end{equation}
I'm confused about what he means,
\begin{quotation}
    These transformation rules incorporate most of what is meant when we say that P or T are 'conserved'.
\end{quotation}

Later, he points out that the above equations of P and T are merely approximation.
These are provided by T. D. Lee, C. N. Yang and others works\cite{PhysRev.104.254}\cite{PhysRevLett.13.138}. 
Before reading the references, I think that the problem arises from that equation (1.0.3) that physicists originally define is wrong, or merely an approximation.
Still, I would like to know why and why it is regarded as approximation.
\section{Lippmann-Schwinger equation}
\begin{equation}
    \Psi_{\alpha}^{+} = \Phi_{\alpha}+\left(E(|\mathbf{k}|-H_{0}+i\epsilon)\right)^{-1}V\Psi_{\alpha}^{+}
\end{equation}
It seems Weinberg's reason\cite{Weinberg_2015} of adding the positive infinitesimal parameter $\epsilon$ is weird, and it is unclear and unintuitive for giving meaning to operator.
Although the sentence of giving meaning to operator is unclear, it is also mentioned in Weinberg's another textbook for quantum field theory\cite{Weinberg_1995_3} that this is because of operator $(E_{\alpha}-H_0)$ isn't invertible, and this is because $E_{\alpha}$ is the eigenvalue of $H_0$; however, restricting the infinitesimal parameter to be positive is still confusing.
Possible solution is to re-derive the Lippmann-Schwinger equation\cite{PhysRev.79.469} in path integral formalism.
\section{Meaning and Function of Potential Operator V}
This problem has been asked before\cite{676400}, and there are few books that discussed about it.
When we expand the second term of equation (2.0.1) with free-interaction state $\Phi_{\beta}$, we get
\begin{equation}
    \Psi_{\alpha}^{+} = \Phi_{\alpha} + \int{d\beta\,\frac{(\Phi_{\beta}, V\Psi_{\alpha}^{+})}{(E(|\mathbf{k}|)-E_{\beta}+i\epsilon)}\Phi_{\beta}}
\end{equation}
and define the T matrix, which later will be seen in S matrix,
\begin{equation}
    T_{\beta\alpha}^{+} \equiv (\Phi_{\beta}, V\Psi_{\alpha}^{+})
\end{equation}
Now this is weird to me because it is unclear about the inner product between two different vectors from two different Hilbert spaces\footnote{That is, one is one of the eigenvectors of free-interaction $H_0$ and the other is one of the eigenvectors of total Hamiltonian $H$.}.

\section{Notes for Lippmann-Schwinger Equation Paper}
\subsection{In Out States}
In this section, I would derive Lippmann-Schwinger equation in similar way as S.Weinberg \cite{Weinberg_2015}, but would first derive the formula of S-matrix and T-matrix.
We define 'in' and 'out' states $\Psi_{\alpha}^{+}$ and $\Psi_{\beta}^{-}$ as eigenstates of the total Hamiltonian
\begin{equation}
    H\Psi_{\alpha}^{\pm} = E_{\alpha}\Psi_{\alpha}^{\pm}
\end{equation}
that both states look like an eigenstate $\Phi_{\alpha}$ of the free-particle Hamiltonian
\begin{equation}
    H_0\Phi_{\alpha} = E_{\alpha}\Phi_{\alpha}
\end{equation}
Note that we describe these states in Schr$\ddot{\text{o}}$diger picture, and this means we must use wave packet to describe the behavior of particle because if not we only got a phase with same state when acting time-evolution operator on state $\exp(-iHt)\Psi_{\alpha}^{\pm} = \exp(-iE_{\alpha}t)\Psi_{\alpha}^{\pm}$
\begin{equation}
    \Psi_{g}^{\pm}(t) = \int{d\alpha\,g(\alpha)\exp(-iE_{\alpha}t)\Psi_{\alpha}^{+}}
\end{equation}
where amplitude $g(\alpha)$ is a smooth-varying function when time evolves.
We further require the 'in' and 'out' states to satisfy the condition
\begin{equation}
    \Psi_{g}^{\pm}(t)\rightarrow \int{d\alpha\,g(\alpha)\exp(-iE_{\alpha}t)\Phi_{\alpha}}
\end{equation}
when $t\rightarrow-\infty$ for 'in' state and $t\rightarrow\infty$ for 'out' state.
Written equation (4.1.3) and (4.1.4) in 
\begin{equation}
    \begin{split}
        \Psi_{g}^{\pm}(t) = \exp(-iHt)\int{d\alpha\,g(\alpha)\Psi_{\alpha}^{\pm}}\\
        \Phi_{g}(t) = \exp(-iH_0t)\int{d\alpha\,g(\alpha)\Phi_{\alpha}}
    \end{split}
\end{equation}
With condition (4.1.4) we could find out the relation
\begin{equation}
    \Psi_{\alpha}^{\pm} = \Omega(\mp\infty)\Phi_{\alpha}
\end{equation}
where $\Omega(t) = \exp(+iHt)\exp(-iH_0t)$
\subsection{S-matrix}
Equation (4.1.1) implies the 'in' and 'out' states should inhabit same Hilbert space, so we can express the 'out' state as linear combination of 'in' states, and the coefficients are defined as S-matrix.
S-matrix will completely contain the information of scattering
\begin{equation}
    S_{\beta\alpha} = \left(\Psi_{\beta}^{-}, \Psi_{\alpha}^{+}\right)
\end{equation}
we could further express S-matrix as by equation (4.1.6)
\begin{equation}
    S_{\beta\alpha} = \left(\Phi_{\beta}, U(+\infty, -\infty)\Phi_{\alpha}\right)
\end{equation}
where $U(t, t_0) = \Omega^{\dagger}(t)\Omega(t_0) = \exp(iH_0t)\exp(-iH(t-t_0))\exp(-iH_0t_0)$.
and define the collision operator such that 
\begin{equation}
    \left(\Phi_{\beta}, S\Phi_{\alpha}\right) \equiv S_{\beta\alpha}
\end{equation}
Then collision operator would be
\begin{equation}
    S=U(+\infty, -\infty)
\end{equation}

Now focus on the time-evolution operator $U(t, -\infty)$, it satisfies the initial condition and differential equation
\begin{equation}
    \begin{split}
        U(-\infty, -\infty) = 1\\
        i\frac{\partial}{\partial t}U(t, -\infty) =V(t)U(t, -\infty)
    \end{split}
\end{equation}
with $v(t) = \exp(iH_0t)V\exp(-iH_0t)$
The solution would be
\begin{equation}
    U(t, -\infty) = 1 - i\int_{-\infty}^{t}{d\tau\, V(\tau)\,U(\tau, -\infty)}
\end{equation}
Changing the upper bound of the integral,
\[ = 1 - i\int_{-\infty}^{\infty}{d\tau\,\theta(t-\tau)V(\tau)U(\tau,-\infty)}\]
The collision matrix would be
\begin{equation}
    S = 1 - i\int_{-\infty}^{\infty}{dt\,V(t)U(t,-\infty)}
\end{equation}
We could consider the part of the operator that generates the interaction
\begin{equation}
    \begin{split}
        T = S - 1\\
        T = -i\int_{-\infty}^{\infty}{dt\,V(t)U(t,-\infty)}
    \end{split}
\end{equation}
and the coefficient would be
\begin{equation}
    T_{\beta\alpha} = -i\int_{-\infty}^{\infty}{dt\,\left(\Phi_{\beta},V(t)U(t,-\infty)\Phi_{\alpha}\right)}
\end{equation}
Expand the interaction $V(t) = \exp(iH_0t)V\exp(-iH_0t)$,
\[T_{\beta\alpha} = -i\int_{-\infty}^{\infty}{dt\,\left(e^{-iH_0t}\Phi_{\beta},Ve^{-iH_0t}U(t, -\infty)\Phi_{\alpha}\right)}\]
\begin{equation} = -i\left(\Phi_{\beta}, V\Psi_{\alpha}(E_{\beta})\right)\end{equation}

Practically, we cannot anticipate to observe Hilbert space that corresponds to free Hamiltonian, we cannot really construct eigenstates $\Phi_{\alpha}$ of free Hamiltonian without interaction.
Schwinger suggests considering equivalent description by simulating the cessation of interaction, and the interaction is regarded to arise from the separation with the free Hamiltonian of the system.
The cessation of the interaction could be done by adding an adiabatic decrease in the interaction strength as $t\rightarrow\pm\infty$.
This could be represented by the factor $\exp(-\epsilon\left|t\right|)$, where $\epsilon$ is a positive infinitesimal parameter.
Thus we consider $\Phi_{\beta}$ to be the exact eigenstate of free Hamiltonian under this condition
Then in equation (4.2.10)
\begin{equation}
    \Psi_{\alpha}(E_{\beta}) = \int_{-\infty}^{\infty}{dt\,e^{i(E_{\beta}-H_0)t}e^{-\epsilon\left|t\right|}U(t,-\infty)\Phi_{\alpha}}
\end{equation}
Taking equation (4.2.6)
\[=\int_{-\infty}^{\infty}{dt\,e^{i(E_{\beta}-E_{\alpha})t}e^{-\epsilon|t|}\Phi_{\alpha}}\]
\[ -i \int_{-\infty}^{\infty}{dt\,e^{i(E_{\beta}-H_0)t}e^{-\epsilon|t|}}\]
\[\times\int_{-\infty}^{\infty}{dt'\,\theta(t-t')e^{iH_0t'}Ve^{-iH_0t'}U(t',-\infty)\Phi_{\alpha}}\]
adding $e^{-iE_{\beta}t'}e^{iE_{\beta}t'}$ between $V$ and $e^{-iH_0t'}$, the second term would be
\[-i\int_{-\infty}^{\infty}{dt\,e^{i(E_{\beta}-H_0)t}e^{-\epsilon|t|}}\]
\[\times\int_{-\infty}^{t}{dt'\,e^{\epsilon |t'|}e^{-\epsilon|t'|}e^{iH_0t'}Ve^{-iE_{\beta}t'}e^{iE_{\beta}t'}e^{-iH_0t'}U(t', -\infty)\Phi_{\alpha}}\]
and then change the integral range
\begin{equation}
    \int_{-\infty}^{\infty}{dt}\int_{-\infty}^{t}{dt'}\rightarrow\int_{-\infty}^{\infty}{dt'}\int_{t'}^{\infty}{dt}
\end{equation}
Plug in the definition (4.2.11) and change the integral variable $t\rightarrow t-t'$
\[\Psi_{\alpha}(E_{\beta}) = \int_{-\infty}^{\infty}{dt\,e^{i(E_{\beta}-E_{\alpha})t}e^{-\epsilon|t|}\Phi_{\alpha}}\]
\begin{equation}
    -i\int^{\infty}_{0}{d\tau\,e^{i(E_{\beta}-H_0)\tau}e^{-\epsilon\tau}V\Psi_{\alpha}(E_{\beta})}
\end{equation}

The second term is easy to evaluate
\begin{equation}
    -i\int_{0}^{\infty}{d\tau\,e^{i(E_{\beta}+i\epsilon-H_0)\tau}} = \frac{1}{E_{\beta}+i\epsilon-H_0}
\end{equation}
and the first term would be a little tricky, that is, it could be regarded as Fourier transform of Dirac delta function, so equation (4.2.13) becomes
\begin{equation}
    \Psi_{\alpha}(E_{\beta}) = 2\pi\delta(E_{\beta}-E_{\alpha})\Phi_{\alpha}+\frac{1}{E_{\beta}+i\epsilon-H_0}V\Psi_{\alpha}(E_{\beta})
\end{equation}

We could further simplify equation (4.2.15) by rewriting equation (4.2.11)
\[\Psi_{\alpha}(E_{\beta}) = \int_{-\infty}^{\infty}{dt\,e^{i(E_{\beta}-H_0)t}e^{-\epsilon|t|}\Omega^{\dagger}(t)\Omega(-\infty)\Phi_{\alpha}}\]
\[= \int_{-\infty}^{\infty}{dt\,e^{i(E_{\beta}-H_0)t}e^{-\epsilon|t|}e^{iH_0t}e^{-iHt}\Psi_{\alpha}}\]
\[ = \int_{-\infty}^{\infty}{dt\,e^{iE_{\beta}}e^{-iE_{\alpha}t}\Psi_{\alpha}}\]
\[ = 2\pi\delta(E_{\beta}-E_{\alpha})\Psi_{\alpha}\]
Therefore, equation (4.2.15) becomes
\begin{equation}
    \Psi_{\alpha} = \Phi_{\alpha} + \frac{1}{E_{\beta}+i\epsilon-H_0}V\Psi_{\alpha}
\end{equation}
which is Lippmann-Schwinger equation.

As we discussed above, $\epsilon$ is a positive infinitesimal parameter, and this solved my previous confusion about the physical meaning of $\epsilon$\footnote{Weinberg only states that $\epsilon$ is added because the operator $(E_{\beta}-H_0)$ is non-invertible.}.
\bibliographystyle{plain}
\bibliography{refs}
\end{document}