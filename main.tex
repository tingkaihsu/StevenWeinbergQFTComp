\documentclass[12pt]{article}
\usepackage{braket}
\usepackage{physics}
\usepackage{graphicx}
\usepackage{times}
\usepackage[export]{adjustbox}
\usepackage{listings}
\usepackage{mathcomp}
\usepackage{hyperref}
\usepackage{bm,amsmath}
\usepackage{float}
\usepackage{indentfirst}
\usepackage{bigints}
\usepackage{listings}
\usepackage{color}
\hypersetup{
colorlinks=true,
linkcolor=blue,
filecolor=magenta,
urlcolor=cyan,
pdftitle={Overleaf Example},
pdfpagemode=FullScreen,
}
\definecolor{dkgreen}{rgb}{0,0.6,0}
\definecolor{gray}{rgb}{0.5,0.5,0.5}
\definecolor{mauve}{rgb}{0.58,0,0.82}
\lstset{frame=tb,
language=Python,
aboveskip=3mm,
belowskip=3mm,
stepnumber = 1,
showstringspaces=false,
columns=flexible,
basicstyle={\small\ttfamily},
numbers=left,
numberstyle=\color{gray},
keywordstyle=\color{blue},
commentstyle=\color{dkgreen},
stringstyle=\color{mauve},
breaklines=true,
breakatwhitespace=true,
tabsize=3
}
\numberwithin{equation}{subsection}

\title{Quantum Field Theory Problems}
\author{Ting-Kai Hsu}
\date{\today}

\begin{document}
\maketitle
\tableofcontents
\section{T P Symmetries}
In S.Weinberg famous textbook about quantum field theory\cite{Weinberg_1995} section 2.6, he discusses parity$\mathcal{P}$ and time inversion$\mathcal{T}$. 
\footnote{I use different notation with Weinberg.}
\begin{equation}
    \begin{split}
        \mathcal{P}^{\mu}_{\,\,\,\,\nu} = \begin{pmatrix}
            1&0&0&0\\
            0&-1&0&0\\
            0&0&-1&0\\
            0&0&0&-1\\
        \end{pmatrix}\\
        \mathcal{T}_{\,\,\,\,\nu}^{\mu} = \begin{pmatrix}
            -1&0&0&0\\
            0&1&0&0\\
            0&0&1&0\\
            0&0&0&1\\
        \end{pmatrix}
    \end{split}
\end{equation}
The operators of $\mathcal{P}$ and $\mathcal{T}$ are believed to be
\begin{equation}
    \begin{split}
        \text{P} \equiv U(\mathcal{P}, 0)\\
        \text{T} \equiv U(\mathcal{T}, 0)
    \end{split}
\end{equation}
He seems to define a new notation that corresponds to not only Lorentz transformation and translation but also parity and time inversion.
The operators of Poincar\'e algebra would transform according to the following law,
\begin{equation}
    \begin{split}
        \text{P}U(\Lambda, a)\text{P}^{-1} = U(\mathcal{P}\Lambda\mathcal{P}^{-1}, \mathcal{P}a)\\
        \text{T}U(\Lambda, a)\text{T}^{-1} = U(\mathcal{T}\Lambda\mathcal{T}^{-1}, \mathcal{T}a)
    \end{split}
\end{equation}
I'm confused about what he means,
\begin{quotation}
    These transformation rules incorporate most of what is meant when we say that P or T are 'conserved'.
\end{quotation}

Later, he points out that the above equations of P and T are merely approximation.
These are provided by T. D. Lee, C. N. Yang and others works\cite{PhysRev.104.254}\cite{PhysRevLett.13.138}. 
Before reading the references, I think that the problem arises from that equation (1.0.3) that physicists originally define is wrong, or merely an approximation.
Still, I would like to know why and why it is regarded as approximation.
\bibliographystyle{plain}
\bibliography{refs}
\end{document}